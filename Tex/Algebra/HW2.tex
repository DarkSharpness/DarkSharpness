\documentclass[fontset=windows]{article}
\usepackage[margin=1in]{geometry}%设置边距,符合Word设定
\usepackage{ctex}
\usepackage{amsmath, xparse}
\usepackage{amssymb}
\usepackage{setspace}
\usepackage{lipsum}
\usepackage{graphicx}%插入图片
\graphicspath{{Figures/}}%文章所用图片在当前目录下的 Figures目录
\usepackage{hyperref} % 对目录生成链接,注:该宏包可能与其他宏包冲突,故放在所有引用的宏包之后
\hypersetup{
    colorlinks        = true,  % 将链接文字带颜色
	bookmarksopen     = true, % 展开书签
	bookmarksnumbered = true, % 书签带章节编号
	pdftitle          = 标题, % 标题
	pdfauthor         = DarkSharpness, % 作者
    citecolor         = green,
    linkcolor         = red,
    urlcolor          = blue}
\bibliographystyle{plain}% 参考文献引用格式
\newcommand{\upcite}[1]{\textsuperscript{\cite{#1}}}

% \renewcommand{\contentsname}{\centerline{Contents}} %经过设置word格式后,将目录标题居中

% Keywords command
\providecommand{\keywords}[1]
{
  \textbf{\text{Keywords: }} #1
}

\title{\heiti\zihao{2} Homework 01}
\author{DarkSharpness}
\date{2023.09.21}

\begin{document}
	\maketitle
	% \thispagestyle{empty}
    \tableofcontents


\section*{T5}

由题,因为交换群,所以 $\forall a,b \in G $, $(a \circ b) ^ m = (a ^ m) \circ (b ^ m) = e e = e$, 因此 $a \circ b \in G$ 。对于任意的 $a \in G$, $a ^ m = e$,  $e = {a \circ a ^ {-1}} ^ m = a ^ m \circ (a ^ {-1}) ^ m = e \circ (a ^ {-1}) ^ m = (a ^ {-1}) ^ m$. 所以 $a ^ {-1} \in G$ 。因此 $H$ 是 $G$ 的子群。

\section*{T6}

$\forall a,b \in H$, 设 $a = gXg^{-1}$, $b = gYg^{-1}$ , 则 $ab = gXg^{-1}gYg^{-1} = gXYg^{-1} \in H$ 。且 $ a ^ {-1} = g ^ {-1} X ^ {-1} g = gX ^ {-1} g ^ {-1} \in H$ 。因此 $H$ 是 $G$ 的子群。

注: 由群的性质,显然 $X^{-1}$, $XY$ 也在群内。


\section*{T7}

$\forall x,y \in C(a)$, $xya = xay = axy$ ,因此 $xy \in C(a)$ 。同时, $ x ^{-1} a ^{-1} = (ax)^{-1} = (xa)^{-1} = a^{-1}x^{-1}$ ,因此(左右分别乘以 $a$ 之后) $a x^{-1} = x^{-1} a$ ,因此 $x^{-1} \in C(a)$ 。

\section*{T8}

$C(G) = \{g \in G | gx = xg, \forall x \in G\}$ 因此显然,$\forall a \in G$ , $ C(G) \subseteq C(a)$ ,因此显然 $C(G) \subseteq \bigcap\limits_{a\in G} C(a)$ 。而由定义, $\forall x \in \bigcap\limits_{a\in G} C(a)
$ , $\forall a \in G , xa = ax$ ,因此 $x \in C(G)$ ,所以 $\bigcap\limits_{a\in G} C(a) \subseteq C(G)$ 。

综上 $\bigcap\limits_{a\in G} C(a) = C(G)$ 。

\section*{T18}

在整数加群中, $\forall a \in \langle m \rangle,b \in \langle n \rangle \exists x,y$ ,满足 $\forall a = xm, y = bn$ 。记 $m_0 = \frac{m}{d} , n_1 = \frac{n}{d} $ ,则 $a + b = xm + yn = d(xm_0 + yn_0) \in \langle d \rangle$ 。因此 $\langle m,n \rangle \subseteq \langle d \rangle$ 。而因为 $(m_0,n_0) = 1$ ,所以当 $xm_0 + yn_0 = 1$ 可以取遍 $\mathbb{Z}$ 。因此 $\forall x \in \langle d \rangle$ ,必定满足 $x \in \langle m,n \rangle$ 。因此 $\langle d \rangle \subseteq \langle m,n \rangle$ 。因此 $\langle m,n \rangle = \langle d \rangle$ 。


\section*{T19}

充分性显然。下证明其必要性。若 $\langle m \rangle = \langle n \rangle$ ,考虑群中绝对值非零且最小的一项,分别为 $|m|$ 和 $|n|$ 。因为 $\langle m \rangle = \langle n \rangle$ ,所以 $|m| = |n|$ ,所以 $m = \pm n$ 。

\section*{补充}

如果 $N = n_1 \cdot n_2$ 且 $gcd(n1,n2) = 1$ 。则 
$$\mathbb{Z}_N^{*} = \mathbb{Z}_{n_1}^{*} \times \mathbb{Z}_{n_2}^{*}$$

证明: $\mathbb{Z}_N^{*} = \{1,2,\dots,N -1\}$。
而  $\mathbb{Z}_{n_1}^{*} \times \mathbb{Z}_{n_2}^{*} = \{d | xn_1 + yn_2 \equiv d (mod N) , 0 < d < N ,  \forall x,y\} $ 。

因为 $ (n_1,n_2) = 1 $ ,故存在 $x,y$ 使得 $xn_1 + yn_2 \equiv 1 (mod N)$ 。因此 $d$ 可以取遍 $1,2,\dots,N-1$ ,因此 $$\mathbb{Z}_N^{*} = \mathbb{Z}_{n_1}^{*} \times \mathbb{Z}_{n_2}^{*}$$



\end{document}

\documentclass[fontset=windows]{article}
\usepackage[margin=1in]{geometry}%设置边距,符合Word设定
\usepackage{ctex}
\usepackage{amsmath, xparse}
\usepackage{amssymb}
\usepackage{setspace}
\usepackage{lipsum}
\usepackage{graphicx}%插入图片
\graphicspath{{Figures/}}%文章所用图片在当前目录下的 Figures目录
\usepackage{hyperref} % 对目录生成链接,注:该宏包可能与其他宏包冲突,故放在所有引用的宏包之后
\hypersetup{
    colorlinks        = true,  % 将链接文字带颜色
	bookmarksopen     = true, % 展开书签
	bookmarksnumbered = true, % 书签带章节编号
	pdftitle          = 标题, % 标题
	pdfauthor         = DarkSharpness, % 作者
    citecolor         = green,
    linkcolor         = red,
    urlcolor          = blue}
\bibliographystyle{plain}% 参考文献引用格式
\newcommand{\upcite}[1]{\textsuperscript{\cite{#1}}}

% \renewcommand{\contentsname}{\centerline{Contents}} %经过设置word格式后,将目录标题居中

% Keywords command
\providecommand{\keywords}[1]
{
  \textbf{\text{Keywords: }} #1
}

\title{\heiti\zihao{2} Homework 01}
\author{DarkSharpness}
\date{2023.09.21}

\begin{document}
	\maketitle
	% \thispagestyle{empty}

\tableofcontents


\section*{T4}

$\forall a,b,c \in A$ , 显然 $\phi (a) = \phi(a)$ 即 $ a \sim a $ 。同时,若 $\phi(a) = \phi(b)$ ,则 $\phi(b) = \phi(a)$ ,即 $ a \sim b \rightarrow b \sim a $ 。最后,若 $\phi(a) = \phi(b)$ 且 $\phi(b) = \phi(c)$ ,则 $\phi(a) = \phi(c)$ ,即 $ a \sim b \wedge b \sim c \rightarrow a \sim c $ 。综上, $\sim$ 是等价关系。

因为 $\phi$ 是集合 A 到 B 的映射,所以等价类为 $ [x] = \{a | \phi(a) = x , a \in A \} (x \in B) $

\section*{T8}

$\forall (a,b) \in S$ ,显然 $ab = ba = ab$ ,所以 $(a,b) \sim (a,b)$ 。同时,若 $(a,b) \sim (c,d)$ ,则 $ad = bc$ 即 $bc = ad$ ,即 $ (a,b) \sim (c,d) \rightarrow (c,d) \sim (a,b) $ 。最后,若 $(a,b) \sim (c,d)$ 且 $(c,d) \sim (e,f)$ ,则 $ad = bc \wedge cf = de$ 。因为 $b,d,f \ne 0$,所以 $\frac a b = \frac c d = \frac e f$ ,因此,$af = be$ ,即 $ (a,b) \sim (c,d) \wedge (c,d) \sim (e,f) \rightarrow (a,b) \sim (e,f) $ 。综上, $\sim$ 是等价关系。


\section*{T5}

$ a \oplus b = a + b - 2 , \forall a,b \in \mathbb{Z}$ 。下证 $(\mathbb{Z} , \oplus)$ 构成群。

\begin{enumerate}
    \item 封闭性:$\forall a,b \in \mathbb{Z}$ ,显然 $a \oplus b = a + b - 2 \in \mathbb{Z}$ 。
    \item 结合律:$\forall a,b,c \in \mathbb{Z}$ ,显然 $a \oplus (b \oplus c) = a \oplus (b + c - 2) = a + b + c - 4 = (a + b - 2) + c - 2 = (a \oplus b) \oplus c$ 
    \item 单位元:$e = 2$ ,显然 $a \oplus e = a + e - 2 = a$ 。
    \item 逆元:$\forall a \in \mathbb{Z}$ ,显然 $a \oplus (4 - a) = a + 4 - a - 2 = 2 = e$ 。逆元为 $4 - a$ 。
\end{enumerate}

\section*{T12}

因为 $\forall x \in G$ , $x ^ 2 = e$ ,由逆元定义有 $x = x ^ {-1}$ 。因此,$x y = (xy) ^ {-1} = e$ ,而 $x y y ^ {-1} x ^ {-1} = x e x ^ {-1} = x x ^ {-1} = e$ ,所以 $(xy) ^ {-1} = y ^ {-1} x ^ {-1} = y x$ ,因此 $xy = yx$ 。综上,$G$ 是交换群。

\section*{T13}

必要性,若 $G$ 为交换群, 则 $(ab)^2 = abab = aabb = a^2 b^2$ 。

充分性,若 $(ab)^2 = a^2 b^2$ ,则 $abab = aabb$ ,因此,$a ^{-1} abab b ^{-1} = a ^{-1} aabb b ^{-1}$ ,即 $ba = ab$ 。因此,$G$ 为交换群。 

\section*{T15}

若 $G$ 为有限群,下证明 $x ^ 3 = e $ 的元素个数是奇数 。

首先 $e$ 显然满足。下证 $x ^ 3 = e (x\ne e)$ 的元素个数为偶数。

因为 $x ^ 3 = e$ ,所以 $x ^ 2 = x ^ {-1}$ 。对于任意满足的解 $x$ ,存在 $x ^ {-1}$ 满足 $(x ^ {-1}) ^ 3 = (x ^ 2) ^ 3 = x ^ 6 = ee = e$ 。若 $x = x^{-1}$ ,则 $x ^ 2 = e$ 。又因为 $x ^ 3 = e$ ,所以显然 $x = e$ 与假设矛盾。 所以对于任意 $x \ne e$ ,若 $x ^ 3 = e$ ,则存在唯一对应的 $x ^ {-1} \ne x$ 满足 $(x ^ {-1}) ^ 3 = e$ 也是满足的解,且 $(x ^ {-1}) ^ {-1} = x$ ,即满足的解成对出现,形如 $(x,x^{-1})$ 。又因为是有限群,所以对数个数有限,所以 $x ^ 3 = e (x \ne e)$ 的元素个数为偶数。

综上,$x ^ 3 = e$ 的元素个数是奇数。

\end{document}

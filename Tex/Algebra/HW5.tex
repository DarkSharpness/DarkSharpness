\documentclass[fontset=windows]{article}
\usepackage[margin=1.5in]{geometry}%设置边距,符合Word设定
\usepackage{ctex}
\usepackage{amsmath, xparse}
\usepackage{amssymb}
\usepackage{setspace}
\usepackage{lipsum}
\usepackage{graphicx}%插入图片
\graphicspath{{Figures/}}%文章所用图片在当前目录下的 Figures目录
\usepackage{hyperref} % 对目录生成链接,注:该宏包可能与其他宏包冲突,故放在所有引用的宏包之后
\hypersetup{
    colorlinks        = true,  % 将链接文字带颜色
	bookmarksopen     = true, % 展开书签
	bookmarksnumbered = true, % 书签带章节编号
	pdftitle          = 标题, % 标题
	pdfauthor         = DarkSharpness, % 作者
    citecolor         = green,
    linkcolor         = red,
    urlcolor          = blue}
\bibliographystyle{plain}% 参考文献引用格式
\newcommand{\upcite}[1]{\textsuperscript{\cite{#1}}}

% \renewcommand{\contentsname}{\centerline{Contents}} %经过设置word格式后,将目录标题居中

% Keywords command
\providecommand{\keywords}[1]
{
  \textbf{\text{Keywords: }} #1
}

\title{\heiti\zihao{2} Homework 05}
\author{DarkSharpness}
\date{2023.10.18}

\begin{document}
	\maketitle
	% \thispagestyle{empty}
    \tableofcontents

\section*{T1}

$H=\{(1),(12)(34),(13)(24),(14)(23)\}$ 在 $A_4$ 中的左陪集为:

$$\{(1),(12)(34),(13)(24),(14)(23)\}$$
$$\{(132),(243),(142),(134)\}$$
$$\{(123),(143),(234),(124)\}$$

在 $S_4$ 中的左陪集为:

$$\{(1),(12)(34),(13)(24),(14)(23)\}$$
$$\{(132),(243),(142),(134)\}$$
$$\{(123),(143),(234),(124)\}$$
$$\{(12),(34),(1423),(1324)\}$$
$$\{(13),(24),(1432),(1234)\}$$
$$\{(14),(23),(1243),(1342)\}$$

\section*{T8}

两个。 $\langle a^2 \rangle$ 和 $a \langle a^2 \rangle = \{a^n | n = 2k - 1, k \in [15]\}$

\section*{T11}

设子群为 $H \le G$ ,设左陪集为 $aH, a \in G$ 。

$\forall x \in aH, x = ah$ ,因此 $x^{-1} = h^{-1}a^{-1} \in Ha^{-1}$ ,即 $aH$ 中的元素的逆元素在 $Ha^{-1}$ 中。同理,任意 $x \in Ha^{-1}, x = ha^{-1}$ ,因此,$x^{-1} = a^{-1}h^{-1} \in aH$ ,即 $Ha^{-1}$ 每个元素都是 $aH$ 中某个元素的逆元素。所以可知, $aH$ 所有元素的逆元素组成了这个子群的右陪集 $Ha^{-1}$ 。

\section*{T12}

$$
\begin{aligned}
    x \in a(H_1 \cap H_2)
    \iff & \exists h, h \in H_1 \cap H_2 \wedge x = ah \\
    \iff & \exists h, h \in H_1 \wedge h \in H_2 \wedge x = ah \\
    \iff & \exists h_1,h_2, h_1 \in H_1 \wedge h_2 \in H_2\\
    & \wedge x = ah_1 \wedge x = ah_2 (\text{a.k.a} ~\ h_1 = h_2 ~\ \text{in this case})\\
    \iff &x \in aH_1 \wedge x \in aH_2 \\
    \iff &x \in aH_1 \cap aH_2
\end{aligned}
$$

\section*{T20}

设 $|G| = 3$ 。构造 $G^* = \{(a,b,c) | abc = 1~\ , a,b,c \in G\}$ 。因为确定了 $a,b$ 之后,存在唯一的 $c \in G$ 满足条件,所以 $|G^*| = (3n)^{3 - 1} = 9n^2$ 。定义 $A = \{(a,a,a) | a^3 = 1~\ , a \in G\}$ , $B = \{(a,b,c) | abc = 1 ~\ , a \ne b \vee b \ne c \vee c \ne a ~\ , a,b,c \in G\}$ 。显然, $|G^*| = |A| + |B|$ 。对于任意 $\{a,b,c\} \in B$ , $(a,b,c),(b,c,a),(c,a,b) \in B$ 。因此,对于每个 $B$ 中元素,存在唯一与之对应的两组元素,即 $B$ 可以看作由一些无序三元组中的每个元素构成。设 $\exists m \in \mathbb{Z},|B| = 3m $ ,又因为 $|G^*| = |A| + |B|$ ,因此可知 $\exists k \in \mathbb{Z},|A| = 3k$ 。

注意到: $(1,1,1) \in A$ ,因此 $|A| \ge 3$ ,即 $\exists g \in G, (g,g,g) \in A, g \ne 1$ ,即 $G$ 中存在 $3$ 阶元素.

注: 本结论为 Cauchy 定理的特例。

\section*{22}

因为交换群,所以: $\pi: a \mapsto a^n$ 满足 $\pi(ab) = (ab)^n = a^n b^n = \pi(a) \pi(b)$ ,即 $\pi$ 为 $G$ 的自同态。

下证 $\pi$ 为双射。

若存在 $a,b \in G, a\ne b \wedge \pi(a) = \pi(b)$ ,则由交换群性质, $(ab^{-1})^n = 1$ ,因此 $ord(\langle ab^{-1} \rangle)\ |\ n$ 。因为 $ab^{-1} \in G$ ,因此 $ord(\langle ab^{-1} \rangle)\ |\ ord(G)$ 。而注意到 $(ord(G),n) = 1$ 因此 $ord(ab^{-1}) = 1$ 即 $a=b$ 矛盾。所以 $a,b \in G, a\ne b$ ,有 $\pi(a) \ne \pi(b)$ 即为单射。

因为 $\pi$ 为 $G$ 到自身的单射,且 $G$ 有限,因此由元素数相等可知,为满射。

综上 $\pi$ 为 $G$ 的自同构。

\end{document}

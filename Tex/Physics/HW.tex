\documentclass{article}
\usepackage{CJKutf8}
\usepackage{amsmath}
\usepackage{amssymb}
\title{Physics 6}
\begin{document}
	\begin{CJK}{UTF8}{gbsn}
		\maketitle
		\section{35}
		\subsection{11}
		\subsubsection{(a)}
		设物体的两端点距离镜面的距离分别为p与p+L;\\
		则其像的位置满足$\frac{1}{p}+\frac{1}{i_1}=\frac{1}{f},\frac{1}{p+L}+\frac{1}{i_2}=\frac{1}{f};\\$
		解得$i_1=\frac{fp}{p-f},i_2=\frac{f(p+L)}{p+L-f};\\$
		故像长为$L'=i_1-i_2=\frac{f^2L}{(p-f)(p+L-f)}\approx L(\frac{f}{p-f})^2;$
		\subsubsection{(b)}
		$m=-\frac{i}{p}=-\frac{f}{p-f},m'=\frac{L'}{L}=(\frac{f}{p-f})^2=m^2;$
		\subsection{15}
		设两个折射点分别为A和B,AB中点为C,硬币位于P,像位于V;\\
		$\angle CVB=\theta_2,\angle CPB=\theta_1,BC=x;\\$
		则$\tan\theta_2=\frac{x}{d_a},\tan\theta_1=\frac{x}{d};\\$
		$\frac{\tan\theta_2}{\tan\theta_1}\approx\frac{\sin\theta_2}{\sin\theta_1};\\$
		$\frac{\frac{x}{d_a}}{\frac{x}{d}}\approx\frac{n_1}{n_2};\\$
		$\frac{d}{d_a}\approx n;\\$
		$d_a\approx \frac{d}{n};$
		\subsection{22}
		$i=(\frac{1}{f}-\frac{1}{p})^{-1}=\frac{fp}{p-f};\\$
		$h_i=mh_p=\frac{i}{p}h_p=\frac{fh_p}{p-f}=5.0mm;$
		\subsection{29}
		考虑一个放在无穷远处的物体成的像;\\
		对于第一个透镜,计算得之其成一个像距为$\frac{1}{f_1}$的实像,这个实像相对于第二个透镜是一个像距为$-\frac{1}{f_1}$的虚像;\\
		故$-\frac{1}{f_1}+\frac{1}{f}=\frac{1}{f_2},f=\frac{f_1f_2}{f_1+f_2};$
		\subsection{32}
		$d_{ey}=\frac{d_{ob}}{m_\theta}=2.1mm;$
		\subsection{34}
		\subsubsection{(a)}
		$\theta=\frac{h}{P_n};\\$
		$\theta'=\frac{h}{p},\frac{1}{p}=\frac{1}{f}+\frac{1}{P_n};\\$
		故$m_\theta=\frac{\theta'}{\theta}=1+\frac{25cm}{f};$
		\subsubsection{(b)}
		$p=f,m_\theta=\frac{P_n}{f}=\frac{25cm}{f};$
		\subsubsection{(c)}
		代入得分别为3.5和2.5;
		\subsection{36}
		$\frac{1}{f_{ob}}=\frac{1}{i}+\frac{1}{p},f_{ob}=9.62mm;\\$
		$M=-\frac{s}{f_{ob}}\frac{25cm}{f_{ey}}=-125;$
	\end{CJK}
\end{document}
\documentclass{article}
\usepackage[margin=1.5in]{geometry}%设置边距,符合Word设定
\usepackage{ctex}
\usepackage{amsmath, xparse}
\usepackage{amssymb}
\usepackage{setspace}
\usepackage{lipsum}
\usepackage{graphicx}%插入图片
\graphicspath{{Figures/}}%文章所用图片在当前目录下的 Figures目录
\usepackage{hyperref} % 对目录生成链接,注:该宏包可能与其他宏包冲突,故放在所有引用的宏包之后
\hypersetup{
    colorlinks        = true,  % 将链接文字带颜色
	bookmarksopen     = true, % 展开书签
	bookmarksnumbered = true, % 书签带章节编号
	pdftitle          = 标题, % 标题
	pdfauthor         = DarkSharpness, % 作者
    citecolor         = green,
    linkcolor         = red,
    urlcolor          = blue}
\bibliographystyle{plain}% 参考文献引用格式
\newcommand{\upcite}[1]{\textsuperscript{\cite{#1}}}

% \renewcommand{\contentsname}{\centerline{Contents}} %经过设置word格式后,将目录标题居中

% Keywords command
\providecommand{\keywords}[1]
{
  \textbf{\text{Keywords: }} #1
}

\title{\heiti\zihao{2} Combinatiorics Homework 01}
\author{DarkSharpness}
\date{2023.10.6}

\begin{document}
	\maketitle
	% \thispagestyle{empty}

\tableofcontents


\section*{Problem 1}

当 $n = t^2 -2 (t \in \mathbb{N})$ , $k = \frac{1}{2} (n - \sqrt{n + 2})$ 或 $k = \frac{1}{2} (n - \sqrt{n + 2}) - 1$ 时,${ n \choose {k + 1} } - { n \choose k}$ 取到最大值。

否则,当 $n = \lfloor \frac{1}{2} (n - \sqrt{n + 2}) \rfloor$ 时,取到最大值。


\section*{Problem 2}

$$E(|X-Y|) = \frac{1000}{2^{1000}} {1000 \choose 500}$$

\section*{Problem 3}

右边的组合含义是: 从 $n$ 个元素之间有 $n + 1$ 个空隙,从中任取 $a + b + 1$ 个空隙。而我们转化这个模型: 对于左边的 $a$ 个空隙,我们认为其绑定的是空隙右边的元素;对于右边的 $b$ 个空隙,我们认为其绑定的是空隙左边的元素。对于剩下那个空隙,我们认为其实分割先,把左边和右边的元素分开。

此时,如果我们先枚举中间的空隙,那么其可以把 $n$ 个元素分为左边 $k$ 个和右边 $n - k$ 个。而对于左边的 $k$ 个元素,我们相当于从 $k$ 个元素中选 $a$ 个 。对于右边同理。因此,其等于 ${k \choose a} {n - k \choose b}$ 然后对于 $k$ 求和,即为左边的式子。


\section*{Problem 4}

有 $2n$ 个元素,我们不妨将其两两配对得到 $n$ 个对子。右式相当于从 $2n$ 个元素中选出 $n$ 个的方案数量。而对于每个对子,我们存在三种可能 : 选了 $0 \ or \ 1 \ or \ 2$ 个元素。对于选 $n$ 个物品的方案,我们如果被选择了 $2$ 个元素的对子有 $k$ 个。显然,被选了 $0$ 个元素的对子也有 $k$ 个,而选了 $1$ 个元素的对子有 $n - 2k$ 个。

因此,对于右边的式子,我们先枚举选了 $2$ 个元素的对子,再枚举选了 $0$ 个元素的对子,一共有 ${n \choose 2k} {2k \choose k}$种方案。而对于剩下的选了 $1$ 个元素的对子,每种对子有 $2$ 种选择,因此有 $2^{n - 2k}$ 种方案。右边的式子等价于  ${n \choose 2k} {2k \choose k} 2^{n - 2k}$ 求和,即为左边的式子。


\section*{Problem 5}

\subsection*{(a)}

我们考虑 $P(-x) \times P(x)$ 。因为 $P(x) = \sum_{k = 0}^{2n} {a_k x^k} $ ,且显然 $a_k = a_{2n-k}$ ,因此 $P(-x) = \sum_{k = 0}^{2n} {(-1)^k a_{n-2k} x^k}$ 。考虑 $P(x) \times P(-x)$ 的第 $2n$ 次项 $b_{2k}$ 。显然,

$$b_{2k} =\sum_{k = 0}^{2n} {a_k (-1)^{2n-k} a_k} = \sum_{k = 0}^{2n} {(-1)^k a_k^2 }$$

而我们代入多项式的原始含义:

$$P(x) \times P(-x) = (1 + x^2 + x^4)^n = P(x^2)$$

所以由定义, $b_{2k} = a_k$ 。所以得证。

\subsection*{(b)}

代入 $x = \pm 1$ 。 $P(1) = \sum_{k=0}^{2n}{a_k}$ 且 $P(-1) = \sum_{k=0}^{2n}{(-1)^k a_k}$ 。所以

$$ a_0 + \dots + a_{2n} = \frac{1}{2}(P(1) + P(-1)) = \frac{1}{2}(3^n + (-1)^n) $$

\subsection*{(c)}

考虑 $G(x) = (1 - x) ^ n $ 。设 $G(x) = \sum_{k=0}^{n}{b_k x^k}$ 。显然 $b_k = (-1)^k {n \choose k}$ 。令 $H(x) = G(x) P(x) = (1 - x^3) ^ n$ 。设 $H(x) = \sum_{k=0}^{3n}{c_k x^k}$ 。显然 $c_k = \sum_{i=0}^{k}{b_i a_{k-i}} = \sum_{i=0}^{k}{(-1)^i {n \choose i} a_{k-i}}$ 。所以所求即为 $c_k$ 。显然:

$$
c_k =
\begin{cases}
    (-1)^k {n \choose t} & (k = 3 * t,~\ t \in \mathbb{Z}) \\
    0 & (\text{otherwise})
\end{cases}
$$

\section*{Problem 6}

显然,

$$
\begin{aligned}
\frac{{n \choose b}} {{n \choose a}} 
    &= \frac{a! (n - a)!} {b! (n - b)!} \\
    &= \frac{a! (n - a)! (b - a)!} {b! (n - b)! (b - a)!} \\
    &= \frac{ {{n - a} \choose {n - b}} } { {b \choose {a}} } \\
\end{aligned}
$$

如果 $\gcd ({n \choose b},{n \choose a}) = 1$ ,那么 $\frac{{n \choose b}} {{n \choose a}}$ 一定已经是最简形式的分式,但 ${b \choose a} < { n \choose a} $ 说明其一定不是最简形式的分式,因此矛盾。所以 $\gcd ({n \choose b},{n \choose a}) > 1$ 。



\end{document}

\documentclass[fontset=windows]{article}
\usepackage[margin=1in]{geometry}%设置边距,符合Word设定
\usepackage{ctex}
\usepackage{amsmath, xparse}
\usepackage{amssymb}
\usepackage{setspace}
\usepackage{lipsum}
\usepackage{graphicx}%插入图片
\graphicspath{{Figures/}}%文章所用图片在当前目录下的 Figures目录
\usepackage{hyperref} % 对目录生成链接,注:该宏包可能与其他宏包冲突,故放在所有引用的宏包之后
\hypersetup{
    colorlinks        = true,  % 将链接文字带颜色
	bookmarksopen     = true, % 展开书签
	bookmarksnumbered = true, % 书签带章节编号
	pdftitle          = 标题, % 标题
	pdfauthor         = DarkSharpness, % 作者
    citecolor         = green,
    linkcolor         = red,
    urlcolor          = blue}
\bibliographystyle{plain}% 参考文献引用格式
\newcommand{\upcite}[1]{\textsuperscript{\cite{#1}}}

\renewcommand{\contentsname}{\centerline{Contents}} %经过设置word格式后,将目录标题居中

% Keywords command
\providecommand{\keywords}[1]
{
  \textbf{\text{Keywords: }} #1
}


\title{\heiti\zihao{2} 场论的运用}
\author{DarkSharpness}
\date{2023.06.06}

\begin{document}
	\maketitle
	% \thispagestyle{empty}

% 摘要
\begin{abstract} 
    这是摘要
\end{abstract}

\keywords{ 场论 , 流体力学 }

\tableofcontents

% 这是图片
% Hello world! Hello Ali! As shown in figure \ref{1}
% \begin{figure}[htbp]
% 	\centering
% 	\includegraphics[scale=0.2]{Ali.jpg}
% 	\caption{this is Ali}
% 	\label{1}
% \end{figure}


\section{引言}

这是引言。


\section{流体运动的基本表述}

在流体力学中,一个重要而基本的问题就是如何描述流体的运动。

\subsection{拉格朗日表述}

该表述的基本思想是,在流体中某一个位置 $\vec{r}_0$ 放置一个标记的试探粒子,其不影响且跟随原来流体运动。此时,通过该粒子的运动的位置矢量 $\vec{r}$ 关于时间的变化关系,即可反映流体运动的特性。因此,我们只需求出 $\vec{r} = \vec{r}(\vec{r}_0,t)$ 即可表征物体的运动。特别地,$\vec{r}(\vec{r}_0,0) = \vec{r}_0$

\subsection{欧拉表述}

该表述的基本思想是,对于某个空间中某个固定的位置 $ \vec{r}_0 $ ,我们去研究该固定点的气体微元的运动状态 $\vec{v} = \vec{v}(\vec{r}_0,t)$,即可。需要特别注意的是,在固定研究空间某一点 $ \vec{r}_0 $ 后,如果为空间中每个流体粒子都打上标记,那么处于 $ \vec{r}_0 $ 的气体微粒的标记可能会随时间而变化。

\subsection{两种表述的差异与联系}

前者给出方程

$$\vec{r} = \vec{r}(\vec{r}_0,t)$$

而后者则是

$$\vec{v} = \vec{v}(\vec{r},t)$$

由物理含义,我们不难得出关系:

$$\vec{v}(\vec{r},t) = \frac{\partial}{\partial t} \vec{r}(\vec{r}_0,t) $$

其中 $\vec{r}_0$ 可以由 $ \vec{r} = \vec{r}(\vec{r}_0,t) $ 反解得到。

由于流体粒子数量众多,且难以区分分辨,在流体力学中,我们往往更多地去用欧拉表述。而拉格朗日表述更加适合于粒子可以辨认,位矢信息比较重要的固体物理中。

值得关注的是,在欧拉表述中,状态量描述的是场的属性,并不直接是粒子的属性。例如关于 $\vec{r}$ 处的气体微元,我们可能用 $f = f(\vec{r},t)$ 的物理量去描述气体。此时, $(\frac{\partial f}{\partial t})_r$ 描述的是在这一点的 f 场的变化,而不是物理量 f 如何随着特定的试探粒子(气体微元)的运动而变化。事实上,在大多物理方程中,例如动量方程 $\vec{F} = \frac{\mathrm{d}\vec{p}}{\mathrm{d}t}$,外界(本例子中为外力)作用的对象是那个运动的气体微元,而不是场。

实际上,常见物理方程中的的 $\frac{\mathrm{d}}{\mathrm{d}t}$ 记号,其是作用于具体粒子。因此其在欧拉表述下,应该是 $(\frac{\partial}{\partial t})_{\vec{r}_0}$ ,即是同一团粒子,而不是简单的 $(\frac{\partial}{\partial t})_{\vec{r}}$ 。为了避免歧义,后文我们将用 $\frac{\mathrm{D}}{\mathrm{D}t}$ 来代替,即:

$$
\frac{\mathrm{D}}{\mathrm{D}t} 
\equiv \frac{\mathrm{d}}{\mathrm{d}t} 
= (\frac{\partial}{\partial t})_{\vec{r}_0}
$$

由链式法则,我们不难求出:

$$
\frac{\mathrm{D}f(\vec{r},t)}{\mathrm{D}t}
= (\frac{\partial f(\vec{r},t)}{\partial t})_{\vec{r}_0}
= (\frac{\partial f}{\partial t})_{\vec{r}}
+ (\frac{\partial f}{\partial r})_t \cdot (\frac{\partial r}{\partial t})_{\vec{r}_0}
$$

在欧拉表述下,我们可以得出,在某一点某一时刻:

$$
\frac{\mathrm{D}}{\mathrm{D}t}
= \frac{\partial}{\partial t} + \vec{v} \cdot \nabla
$$


\section{Reynolds 输运定理}

在流体力学中,物理学家们常常会考察某些特定区域 $ V(t) $ 内的流体状态。

对于一个区域 $ V(t) $ ,边界为 $ \partial V(t) $ ,流体具有某个广度量 $f$ ,很多物理方程中,会出现这部分流体粒子该物理量的总和对于时间的导数。即

$$
\frac{\mathrm{d}}{\mathrm{d}t} \int_{V(t)} f \mathrm{d}V
$$

注意到我们在分析这个式子的时候,我们是对于特定的粒子在分析,并不是背后的场,$\frac{\mathrm{d}}{\mathrm{d}x}$ 必须作用在粒子上。因此, $V$ 会随着时间而改变。

为了避免过多的讨论,这里假设 $ f $ 的性质足够的好,满足函数本身和偏导数都连续,且积分区域为紧集(即有界闭区域)。那么在积分区域上,则有:

$$
\frac{\mathrm{d}}{\mathrm{d}t} \int_{V(t)} f \mathrm{d}V
= \int_{V(t)} \frac{\partial f}{\partial t} \mathrm{d}V +
  \int_{\partial V(t)} {f \vec{v} \cdot \mathrm{d}\vec{S}}
= \int_{V(t)} (\frac{\partial f}{\partial t} + \nabla (f \vec{v})) \mathrm{d}V
$$

\subsection*{物理直观}

从直观上去理解,上述式子的物理含义是在某时刻和某体积内,全部流体某个广度量的总和的变化量。根据物理直观,我们可以猜测: 该变化量等于内部的广度量变化量,以及从边界流出的量。

内部的广度量变化量即等于: 

$$
\int_V \frac{\partial f}{\partial t} \mathrm{d}V
$$

边界流出的量需要定义流。由物理直观,流即为量乘上速度,即: $ \vec{j} \equiv f \cdot \vec{v} $

直观上,边界流出的量等于从所有面积流出的量积分,等于:

$$
\int_{\partial V} \vec{j} \mathrm{d} \vec{S}
$$

因此,合并两项,完全借助物理直观,我们可以得到:

$$
\begin{aligned}
    \int_V \frac{\mathrm{D}f}{\mathrm{D}t} \mathrm{d}V
        &= \int_V \frac{\partial f}{\partial t} \mathrm{d}V +
            \int_{\partial V} \vec{j} \mathrm{d} \vec{S}      \\
        &= \int_V (\frac{\partial f}{\partial t} + \nabla (\rho \vec{v})) \mathrm{d}V
\end{aligned}
$$

其中,最后一步用到了高斯定理。

\subsection*{数学推导}

我们假设体积 $V(t)$ 足够好,其表面方程为 $ F(\vec{r},t) = 0 $ 满足有界性,且 $f$ 在区域内连续且有连续的偏导数。简单起见,我们暂时只证明其一维时候的形式。

$$
\frac{\mathrm{d}}{\mathrm{d}y} \int_{a(y)}^{b(y)}f(x,y) \mathrm{d}x = 
\int_{a(y)}^{b(y)}\frac{\partial f(x,y)}{\partial y} \mathrm{d}x
+ f(b,y) \cdot \frac{\mathrm{d}b}{\mathrm{d}y}
- f(a,y) \cdot \frac{\mathrm{d}a}{\mathrm{d}y}
$$

记 : $ \phi(y) = \int_{a(y)}^{b(y)}f(x,y) \mathrm{d}x $

$$
\phi(y + h) - \phi(y) = \int_{a(y)}^{b(y)}(f(x,y + h) - f(x,y))\mathrm{d}x
                        - \int_{a(y)}^{a(y + h)}f(x,y + h)\mathrm{d}x
                        + \int_{b(y)}^{b(y + h)}f(x,y + h)\mathrm{d}x
$$

由微分中值定理:

$$
\int_{a}^{b}(f(x,y + h) - f(x,y))\mathrm{d}x 
    = h \cdot \int_{a}^{b} \frac{\partial}{\partial y}f(x,y + \theta h) \mathrm{d}x (\theta \in [0,1])
$$

因为连续且有界,所以 $f$ 一致连续,所以

$$
\lim_{h \rightarrow 0} \int_{a}^{b}\frac{f(x,y + h) - f(x,y)}{h}\mathrm{d}x 
    = \int_{a}^{b} \frac{\partial}{\partial y}f(x,y) \mathrm{d}x
$$

所以结合牛顿莱布尼兹公式,可得:

$$
\frac{\mathrm{d}}{\mathrm{d}y} \int_{a(y)}^{b(y)}f(x,y) \mathrm{d}x = 
\int_{a(y)}^{b(y)}\frac{\partial f(x,y)}{\partial y} \mathrm{d}x
+ f(b,y) \cdot \frac{\mathrm{d}b}{\mathrm{d}y}
- f(a,y) \cdot \frac{\mathrm{d}a}{\mathrm{d}y}
$$

对于高维的情况,我们可以类似地证明 \upcite{ref1} ,下式依然成立:


% \section*{这个标题没有数字}

\begin{thebibliography}{30}

    \bibitem{ref1} 维基百科:曲线 \url{https://zh.wikipedia.org/wiki/%E6%9B%B2%E7%BA%BF}

\end{thebibliography}

\end{document}

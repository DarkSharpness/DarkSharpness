\documentclass[fontset=windows]{article}
\usepackage[margin=1in]{geometry}%设置边距,符合Word设定
\usepackage{ctex}
\usepackage{amsmath, xparse}
\usepackage{amssymb}
\usepackage{setspace}
\usepackage{lipsum}
\usepackage{graphicx}%插入图片
\graphicspath{{Figures/}}%文章所用图片在当前目录下的 Figures目录
\usepackage{hyperref} % 对目录生成链接,注:该宏包可能与其他宏包冲突,故放在所有引用的宏包之后
\hypersetup{
    colorlinks        = true,  % 将链接文字带颜色
	bookmarksopen     = true, % 展开书签
	bookmarksnumbered = true, % 书签带章节编号
	pdftitle          = 标题, % 标题
	pdfauthor         = DarkSharpness, % 作者
    citecolor         = red,
    linkcolor         = black,
    urlcolor          = blue
}
\bibliographystyle{plain}% 参考文献引用格式
\newcommand{\upcite}[1]{\textsuperscript{\cite{#1}}}

\renewcommand{\contentsname}{\centerline{Contents}} %经过设置word格式后,将目录标题居中

% Keywords command
\providecommand{\keywords}[1]
{
  \textbf{\text{Keywords: }} #1
}


\title{\heiti\zihao{2} 流体力学与场论}
\author{\href{https://darksharpness.github.io/}{DarkSharpness}}
\date{2023.06.06}

\begin{document}
	\maketitle
	% \thispagestyle{empty}

% 摘要
\begin{abstract} 
    场论是数学中研究函数的一个有力的工具。本文借助场论的知识,以及物理学中基本的质量、动量、能量守恒和一些本构关系,推导出了流体力学中的重要定理、守恒定律,以及某些特殊流体满足的基本方程。
\end{abstract}

\keywords{ 场论 , 流体力学 }

\tableofcontents

\section{引言}

流体力学,是物理学重要的分支,是研究流体运动和力学性质的学科,涵盖了广泛的领域,包括天气预报、空气动力学、海洋学等等。在生活中,我们也常常能见到各种流体,例如水龙头里面流动的水,我们呼吸的空气。而流体也存在许多有趣的现象,例如层流与湍流。在过去,人们一直在不断深化对流体力学的理解,并发展了一系列数学模型和理论框架,来解释这些独特的流体现象。

场论是现代物理学的一个重要分支,它为描述自然界中的基本相互作用提供了强大的数学工具。

本文旨在借助场论的知识,以及物理学中基本的质量、动量、能量守恒和一些本构关系,推导出了流体力学中的重要定理、守恒定律。本文也将讨论,满足特殊条件的流体流体,其基本动力学方程。

\section{流体运动的基本表述}

在流体力学中,一个重要而基本的问题就是如何描述流体的运动。

\subsection{拉格朗日表述}

该表述的基本思想是,在流体中某一个位置 $\vec{r}_0$ 放置一个标记的试探粒子,其不影响且跟随原来流体运动。此时,通过该粒子的运动的位置矢量 $\vec{r}$ 关于时间的变化关系,即可反映流体运动的特性。因此,我们只需求出 $\vec{r} = \vec{r}(\vec{r}_0,t)$ 即可表征物体的运动。特别地,$\vec{r}(\vec{r}_0,0) = \vec{r}_0$

\subsection{欧拉表述}

该表述的基本思想是,对于某个空间中某个固定的位置 $ \vec{r}_0 $ ,我们去研究该固定点的流体微元的运动状态 $\vec{v} = \vec{v}(\vec{r}_0,t)$,即可。需要特别注意的是,在固定研究空间某一点 $ \vec{r}_0 $ 后,如果为空间中每个流体粒子都打上标记,那么处于 $ \vec{r}_0 $ 的流体微粒的标记可能会随时间而变化。

\subsection{两种表述的差异与联系}

前者表述为:

$$\vec{r} = \vec{r}(\vec{r}_0,t)$$

而后者则是:

$$\vec{v} = \vec{v}(\vec{r},t)$$

由物理含义,我们不难得出关系:

$$\vec{v}(\vec{r},t) = \frac{\partial}{\partial t} \vec{r}(\vec{r}_0,t) $$

其中 $\vec{r}_0$ 可以由 $ \vec{r} = \vec{r}(\vec{r}_0,t) $ 反解得到。

由于流体粒子数量众多,且难以区分分辨,在流体力学中,我们往往更多地去用欧拉表述。而拉格朗日表述更加适合于粒子可以辨认,位矢信息比较重要的固体物理中。

值得关注的是,在欧拉表述中,状态量描述的是场的属性,并不直接是粒子的属性。例如关于 $\vec{r}$ 处的流体微元,我们可能用 $f = f(\vec{r},t)$ 的物理量去描述流体。此时, $(\frac{\partial f}{\partial t})_r$ 描述的是在这一点的 f 场的变化,而不是物理量 f 如何随着特定的试探粒子(流体微元)的运动而变化。事实上,在大多物理方程中,例如动量方程 $\vec{F} = \frac{\mathrm{d}\vec{p}}{\mathrm{d}t}$,外界 (本例子中为外力 $\vec{F}$ ) 作用的对象是那个运动的流体微元,而不是场。

实际上,常见物理方程中的的 $\frac{\mathrm{d}}{\mathrm{d}t}$ 记号,其是作用于具体粒子。因此其在欧拉表述下,应该是 $(\frac{\partial}{\partial t})_{\vec{r}_0}$ ,即是同一团粒子,而不是简单的 $(\frac{\partial}{\partial t})_{\vec{r}}$ 。为避免歧义,后文我们将用 $\frac{\mathrm{D}}{\mathrm{D}t}$ 来代替,称为物质导数\upcite{ref0},即:

$$
\frac{\mathrm{D}}{\mathrm{D}t} 
\equiv (\frac{\partial}{\partial t})_{\vec{r}_0}
$$

由链式法则,我们不难求出:

$$
\frac{\mathrm{D}f(\vec{r},t)}{\mathrm{D}t}
= (\frac{\partial f(\vec{r},t)}{\partial t})_{\vec{r}_0}
= (\frac{\partial f}{\partial t})_{\vec{r}}
+ (\frac{\partial f}{\partial r})_t \cdot (\frac{\partial r}{\partial t})_{\vec{r}_0}
$$

在欧拉表述下,我们可以得出,在某一点某一时刻:

$$
\frac{\mathrm{D}}{\mathrm{D}t}
= \frac{\partial}{\partial t} + \vec{v} \cdot \nabla
$$

\section{Reynolds 输运定理}

在流体力学中,物理学家们常常会考察某些特定区域 $ V(t) $ 内的流体状态。

对于一个区域 $ V(t) $ ,边界为 $ \partial V(t) $ ,流体具有某个广度量 $f$ ,很多物理方程中,会出现这部分流体粒子该物理量的总和对于时间的导数。即:

$$
\frac{\mathrm{d}}{\mathrm{d}t} \int_{V(t)} f \mathrm{d}V
$$

注意到我们在分析这个式子的时候,我们是对于特定的粒子在分析,并不是背后的场,$\frac{\mathrm{d}}{\mathrm{d}x}$ 必须作用在粒子上。因此, $V$ 会随着时间而改变。

为了避免过多的讨论,这里假设 $ f $ 的性质足够的好,满足函数本身和偏导数都连续,且积分区域为紧集(即有界闭区域)。那么在积分区域上,则:

$$
\frac{\mathrm{d}}{\mathrm{d}t} \int_{V(t)} f \mathrm{d}V
= \int_{V(t)} \frac{\partial f}{\partial t} \mathrm{d}V +
  \int_{\partial V(t)} {f \vec{v} \cdot \mathrm{d}\vec{S}}
= \int_{V(t)} (\frac{\partial f}{\partial t} + \nabla (f \vec{v})) \mathrm{d}V
$$

该式即为 Reynolds 输运定理 \upcite{ref1}。

\subsection{物理直观}

从直观上去理解,上述式子的物理含义是在某时刻和某体积内,全部流体某个广度量的总和的变化量。根据物理直观,我们可以猜测: 该变化量等于内部的广度量变化量,以及从边界流出的量。

内部的广度量变化量即等于: 

$$
\int_V \frac{\partial f}{\partial t} \mathrm{d}V
$$

边界流出的量需要定义流。由物理直观,流即为量乘上速度,即: $ \vec{j} \equiv f \cdot \vec{v} $

直观上,边界流出的量等于从所有面积流出的量积分,等于:

$$
\int_{\partial V} \vec{j} \mathrm{d} \vec{S}
$$

因此,合并两项,完全借助物理直观,我们可以得到:

$$
\begin{aligned}
    \frac{\mathrm{d}}{\mathrm{d}t} \int_{V(t)} f \mathrm{d}V
        &= \int_V \frac{\partial f}{\partial t} \mathrm{d}V +
            \int_{\partial V} \vec{j} \mathrm{d} \vec{S}      \\
        &= \int_V (\frac{\partial f}{\partial t} + \nabla (\rho \vec{v})) \mathrm{d}V
\end{aligned}
$$

其中,最后一步用到了高斯定理,以及 $\vec{j}$ 的定义。

\subsection{数学推导}

我们假设体积 $V(t)$ 足够好,满足有界性,由分段光滑的闭曲面围成,且 $f$ 在区域内有连续的偏导数,则需证:

$$
\frac{\mathrm{d}}{\mathrm{d}t} \int_{V(t)} f \mathrm{d}V
= \int_{V(t)} \frac{\partial f}{\partial t} \mathrm{d}V +
  \int_{\partial V(t)} {f \vec{v} \cdot \mathrm{d}\vec{S}}
= \int_{V(t)} (\frac{\partial f}{\partial t} + \nabla (f \vec{v})) \mathrm{d}V
$$

简单起见,我们暂时只证明其一维时候的形式。

$$
\frac{\mathrm{d}}{\mathrm{d}y} \int_{a(y)}^{b(y)}f(x,y) \mathrm{d}x = 
\int_{a(y)}^{b(y)}\frac{\partial f(x,y)}{\partial y} \mathrm{d}x
+ f(b,y) \cdot \frac{\mathrm{d}b}{\mathrm{d}y}
- f(a,y) \cdot \frac{\mathrm{d}a}{\mathrm{d}y}
$$

记 : $ \phi(y) = \int_{a(y)}^{b(y)}f(x,y) \mathrm{d}x $

$$
\phi(y + h) - \phi(y) = \int_{a(y)}^{b(y)}(f(x,y + h) - f(x,y))\mathrm{d}x
                        - \int_{a(y)}^{a(y + h)}f(x,y + h)\mathrm{d}x
                        + \int_{b(y)}^{b(y + h)}f(x,y + h)\mathrm{d}x
$$

由微分中值定理:

$$
\int_{a}^{b}(f(x,y + h) - f(x,y))\mathrm{d}x 
    = h \cdot \int_{a}^{b} \frac{\partial}{\partial y}f(x,y + \theta h) \mathrm{d}x (\theta \in [0,1])
$$

因为连续且有界,所以 $f$ 一致连续,所以

$$
\lim_{h \rightarrow 0} \int_{a}^{b}\frac{f(x,y + h) - f(x,y)}{h}\mathrm{d}x 
    = \int_{a}^{b} \frac{\partial}{\partial y}f(x,y) \mathrm{d}x
$$

由牛顿莱布尼兹公式,可得:

$$
\lim_{h \rightarrow 0} \int_{b(y)}^{b(y + h)} \frac{f(x,y + h)}{h}\mathrm{d}x 
    = f(b,y) \cdot \frac{\mathrm{d}b}{\mathrm{d}y}
$$

所以当 $h$ 趋向于 $0$ 的时候,有:

$$
\begin{aligned}
\frac{\mathrm{d}}{\mathrm{d}y} \int_{a(y)}^{b(y)}f(x,y) \mathrm{d}x 
    &= \lim_{h \rightarrow 0} \frac{\phi(y + h) - \phi(y)}{h} \\
    &= \lim_{h \rightarrow 0} \int_{a}^{b}\frac{f(x,y + h) - f(x,y)}{h}\mathrm{d}x +
       \lim_{h \rightarrow 0} \int_{b(y)}^{b(y + h)} \frac{f(x,y + h)}{h}\mathrm{d}x -
       \lim_{h \rightarrow 0} \int_{a(y)}^{a(y + h)} \frac{f(x,y + h)}{h}\mathrm{d}x \\
    &= \int_{a(y)}^{b(y)}\frac{\partial f(x,y)}{\partial y} \mathrm{d}x
+ f(b,y) \cdot \frac{\mathrm{d}b}{\mathrm{d}y}
- f(a,y) \cdot \frac{\mathrm{d}a}{\mathrm{d}y}
\end{aligned}
$$

对于高维的情况,我们可以类似地证明 \upcite{ref2} ,下式依然成立:

$$
\frac{\mathrm{d}}{\mathrm{d}t} \int_{V(t)} f \mathrm{d}V
= \int_{V(t)} \frac{\partial f}{\partial t} \mathrm{d}V +
  \int_{\partial V(t)} {f \vec{v} \cdot \mathrm{d}\vec{S}}
= \int_{V(t)} (\frac{\partial f}{\partial t} + \nabla (f \vec{v})) \mathrm{d}V
$$

在这里,物理直观与数学理论达成了一致。

\section{流体力学的守恒定律}

\subsection{质量守恒定理}

对于固定的一个流体微元,其质量一定是保持不变的,即对任意一些固定的粒子构成的体积:

$$
\frac{\mathrm{d}}{\mathrm{d}t} \int_{V(t)} \rho \mathrm{d}V = 0
$$

由 Reynolds 输运定理,可以得:

$$
\frac{\mathrm{d}}{\mathrm{d}t} \int_{V(t)} 
    (\frac{\partial \rho}{\partial t} + \nabla (\rho \vec{v})) \mathrm{d}V = 0
$$

$$
\frac{\partial \rho}{\partial t} + \nabla (\rho \vec{v}) = 0
$$

其物理含义是: 某一点处的密度增加量,加上该点净流出的所有流体,为零。有时可写作:

$$
\frac{\mathrm{D} \rho}{\mathrm{D} t} + \rho \nabla \cdot \vec{v} = 0
$$


事实上,很多时候,许多的物理量往往正比于流体密度,例如单位体积势能 $\phi(\vec{r},t) = \rho(\vec{r},t) \cdot U(\vec{r},t) $ ,即为密度乘以该点的势(电势/重力势等等)。借助 Reynolds 输运定理,可知:

$$
\begin{aligned}
\frac{\mathrm{d}}{\mathrm{d}t} \int_{V(t)} \phi \mathrm{d}V
    &= \int_{V(t)} (\frac{\partial \phi}{\partial t} + \nabla (\phi \vec{v})) \mathrm{d}V \\
    &= \int_{V(t)} (\frac{\partial \rho}{\partial t} U + \rho \frac{\partial U}{\partial t} 
        + \rho U \nabla \vec{v} + \rho (\nabla U) \vec {v} +  (\nabla \rho) U  \vec{v} ) \mathrm{d}V \\
    &= \int_{V(t)} (U (\frac{\partial \rho}{\partial t} + \nabla (\rho \vec{v}))
        + \rho (\frac{\partial U}{\partial t} + \vec {v} \nabla U ) )\mathrm{d}V \\
    &= \int_{V(t)} \rho \frac{\mathrm{D}U}{\mathrm{D}t} \mathrm{d}V
\end{aligned}
$$

\subsection{动量守恒定理}

由动量守恒定律,作用在一些流体位元的力的大小乘以作用时间,即为动量变化量,因此:

$$
\frac{\mathrm{d}}{\mathrm{d}t} \int_{V(t)} \rho \vec{v} \mathrm{d}V
= \vec{F}
$$

而作用在物体上的外力往往由两部分构成: 一部分是作用在单位质量流体微元上的力,单位体积力大小 $\vec{f}$ 。另一部分是该流体微元表面受到的应力,应力张量为 $\sigma$ 。

因此,由动量守恒定律 :

$$
\frac{\mathrm{d}}{\mathrm{d}t} \int_{V(t)} \rho \vec{v} \mathrm{d}V
= \int_{V(t)} \rho \vec {f} \mathrm{d}V + \int_{\partial V(t)} \sigma \mathrm{d}\vec{S}
$$

由质量守恒定律的推论,以及高斯定理,可得:

$$
\frac{\mathrm{D}\vec{v}}{\mathrm{D}t} = \frac{1}{\rho} \nabla {\sigma} + \vec{f}
$$

$$
\frac{\partial \vec{v}}{\partial t} + (\vec{v} \cdot \nabla) \vec{v} = \frac{1}{\rho} \nabla {\sigma} + \vec{f}
$$

\subsection{能量守恒定理}

流体的总内能由两项组成: 流体的宏观运动动能,流体的微观内能(包含热运动动能,以及分子间势能之和)。一定体积内的总内能即为: $E = \int_V \rho(\frac{v ^ 2}{2} + \epsilon)$ ,其中 $\epsilon$ 为单位体积的微观内能。

特别地,为了简单起见,我们认为热量仅仅通过热传导传递,即热流 $ \vec{j_q} = -\kappa \nabla T$ ,且假设 $\kappa$ 为常数。此时,由热力学第一定律 $\mathrm{d}E = \delta W +  \delta Q$ :

$$
\frac{\mathrm{d}E}{\mathrm{d}t}
= \int_{V(t)} \rho \vec {f} \cdot \vec{v} \mathrm{d}V + \int_{\partial V(t)} \sigma \vec{v} \mathrm{d}\vec{S} + \int_{\partial V(t)} \vec{j_q} \mathrm{d}\vec{S} 
$$

由质量守恒定律、动量守恒定理的推论,以及高斯定理,可得:

$$
\rho(\frac{\partial \epsilon}{\partial t} + (\vec{v} \cdot \nabla) \epsilon) = \nabla (\sigma \vec {v}) - \nabla{\sigma} \cdot \vec{v} - \kappa \nabla^2{T}
$$

\section{流体力学的基本方程}

为了简单起见,在之后的讨论中,我们认为物理过程进行的很快,以至于流体之间完全来不及进行热传导,因此忽略流体之间的热传导,即可以认为 $\kappa = 0$ 。

\subsection{不可压缩黏性流体}

由于不可压缩,因此,任何流体微元在流动过程中,体积不变,只和初始密度有关,即 $\frac{\mathrm{D}\rho}{\mathrm{D}t} = 0$ 。代入质量守恒定律,可得 

$$
\nabla \cdot \vec{v} = 0
$$

同时,我们假设流体性质较好,即满足流体均匀各向同性。由不可压缩流体的本构关系方程 \upcite{ref3} :

$$
\sigma_{i,j} = -p\delta_{i,j} + \mu(\frac{\partial v_i}{\partial x_j} + \frac{\partial v_j}{\partial x_i})
$$

其中 $\delta_{i,j}$ 当且仅当 $i = j$ 的时候为 1 , 否则为 0 。而 $\mu$ 是流体的第一黏性系数。

将本构关系代入动量守恒方程。可得:

$$ 
\frac{\mathrm{D}\vec{v}}{\mathrm{D}t} = -\frac{1}{\rho}\nabla{p} + \frac{\mu}{\rho}\nabla^2{\vec{v}} + \vec{f}
$$

$$ 
\frac{\partial \vec{v}}{\partial t} + (\vec{v} \cdot \nabla) \vec{v}  = -\frac{1}{\rho}\nabla{p} + \frac{\mu}{\rho}\nabla^2{\vec{v}} + \vec{f}
$$

\subsection{可压缩无黏性流体}

当流体没有黏性,此时,应力张量仅仅由压强提供,即 $\sigma = -p I$ ,其中 $I$ 为单位矩阵。

此时,满足:

$$
\begin{aligned}
    \frac{\partial \rho}{\partial t} + \nabla (\rho \vec{v}) &= 0 \\
    \frac{\partial \vec{v}}{\partial t} + (\vec{v} \cdot \nabla) \vec{v} &= - \frac{1}{\rho} \nabla {p} + \vec{f} \\
    \rho(\frac{\partial \epsilon}{\partial t} + (\vec{v} \cdot \nabla) \epsilon) &= -p \nabla \cdot v
\end{aligned}
$$

由 1,3 式:

$$
\frac{\mathrm{D}\epsilon}{\mathrm{D}t} = \frac{p}{\rho} \frac{\mathrm{D}\rho}{\mathrm{D}t}
$$

特别地,当流体为理想气体,满足 $p = \frac{\rho}{\mu} RT$ , $\epsilon = \frac{1}{\mu} C_vT$ ,因此可得: 

$$
\frac{C_v}{R} \frac{\mathrm{D}(\rho T)}{\mathrm{D}t} = T \frac{\mathrm{D}\rho}{\mathrm{D}t}
$$

$$
\frac{\mathrm{D}}{\mathrm{D}t}(\frac{T}{\rho ^ {\gamma - 1}}) = 0
$$

其中 $\gamma = 1 + \frac{R}{C_v} $ 。该式子对应的物理含义为,一个气流团在上升的过程中,保持绝热(即无热交换)。事实上,将 1,3 式直接得到的式子,乘以 $V$ (气流团体积微元的大小),便可以得到:

$$
\frac{\mathrm{D}E}{\mathrm{D}t} = \frac{p}{\rho} \frac{\mathrm{D}\rho}{\mathrm{D}t} V
$$

对于同一团气体,质量始终不变,即 $\frac{\mathrm{D}(\rho V)}{\mathrm{D}t} = 0$,因此可得:

$$
\frac{\mathrm{D}E}{\mathrm{D}t} + p \frac{\mathrm{D}V}{\mathrm{D}t} = 0
$$

即绝热状态下的热力学第一定律 $\mathrm{d}E + p \mathrm{d}V = 0$ 。

\subsection{不可压缩无黏性流体}

由前面可知,此时 $\nabla \cdot \vec{v} = 0$ 且 $\sigma = -p I$ 。

我们再考虑一个特殊的情况: $\vec{f}$ 完全由保守力提供,即 $\vec{f} = -\nabla \Phi$ 且流体密度恒为 $\rho$ 。此时,其方程满足:

$$
\frac{\mathrm{D} \vec{v}}{\mathrm{D} t} = -\frac{1}{\rho}\nabla{p} - \nabla \Phi
$$

将方程两边乘以 $\vec{v}$ ,可得:

$$
\frac{\mathrm{D}}{\mathrm{D}t}(\frac{1}{2}v^2) + \vec{v} \cdot \nabla(\frac{p}{\rho} + \Phi) = 0
$$

当流体满足定常条件,即 $\frac{\partial}{\partial t} = 0$ 。则:

$$
\vec{v} \cdot \nabla(\frac{1}{2} v^2 + \frac{p}{\rho} + \Phi) = 0
$$

因此,在同一条流线上:

$$
\frac{1}{2} \rho v^2 + p + \rho\Phi = const
$$

即为伯努利方程。需要注意的是,伯努利方程成立的条件是不可压缩无黏性流体,且流体满足定常条件,并且仅仅对同一条流线上的流体成立,不同流线的常数可能不同。而一般情况下,我们认为伯努利方程对于全部空间成立,此时需要额外添加无旋条件,即:

$$
\nabla \times \vec{v} = 0
$$

此时,有恒等式:

$$
\nabla(\frac{1}{2} v^2)= \vec{v} \times (\nabla \times \vec{v}) + (\vec{v} \times \nabla) \vec{v} = (\vec{v} \cdot \nabla) \vec{v}
$$

因此,在额外满足了无旋条件后,结合动量守恒方程,我们可以得出: 

$$
\nabla(\frac{1}{2} v^2) = -\frac{1}{\rho}\nabla{p} - \nabla \Phi
$$

$$
\nabla(\frac{1}{2} v^2 + \frac{p}{\rho} + \Phi) = 0
$$

因此,此时,对于全空间,均有 $\frac{1}{2} \rho v^2 + p + \rho\Phi = const$ 。该形式即为我们熟知的伯努利方程。

\section{小结}

在流体力学中,我们往往用场的欧拉表述,来反映流体场的性质。而一般的力学定律中,我们研究的是粒子的属性。因此,我们需要将熟知的定律转化为欧拉表述。由此,我们推导出了物质导数的概念:

$$
\frac{\mathrm{D}}{\mathrm{D}t} = \frac{\partial}{\partial t} + \vec{v} \cdot \nabla
$$

在流体力学中,物理学家们常常会考察某些特定区域 $ V(t) $ 内的流体状态,即:

$$
\frac{\mathrm{d}}{\mathrm{d}t} \int_{V(t)} f \mathrm{d}V
$$

借助物理直观与数学推导,我们可以得出 Reynolds 输运定理:

$$
\frac{\mathrm{d}}{\mathrm{d}t} \int_{V(t)} f \mathrm{d}V
= \int_{V(t)} (\frac{\partial f}{\partial t} + \nabla (f \vec{v})) \mathrm{d}V
$$

通过三大守恒(质量/动量/能量)定律,我们能推导出:

$$
\begin{aligned}
    \frac{\partial \rho}{\partial t} + \nabla (\rho \vec{v}) &= 0 \\
    \frac{\partial \vec{v}}{\partial t} + (\vec{v} \cdot \nabla) \vec{v} &= \frac{1}{\rho} \nabla {\sigma} + \vec{f} \\
    \rho(\frac{\partial \epsilon}{\partial t} + (\vec{v} \cdot \nabla) \epsilon) &= \nabla (\sigma \vec {v}) - \nabla{\sigma} \cdot \vec{v} - \kappa \nabla^2{T}
\end{aligned}
$$

最后,我们推导了绝热情况下,某些特殊流体的基本动力学方程。对于不可压缩黏性流体,其满足:

$$
\begin{aligned}
    \nabla \cdot \vec{v} &= 0 \\
    \frac{\mathrm{D} \vec{v}}{\mathrm{D}t} &= -\frac{1}{\rho}\nabla{p} + \frac{\mu}{\rho}\nabla^2{\vec{v}} + \vec{f}
\end{aligned}
$$

对于可压缩无黏性流体,其满足:

$$
\begin{aligned}
    \frac{\mathrm{D} \vec{v}}{\mathrm{D}t} &= - \frac{1}{\rho} \nabla {p} + \vec{f} \\
    \frac{\mathrm{D}\epsilon}{\mathrm{D}t} &= \frac{p}{\rho} \frac{\mathrm{D}\rho}{\mathrm{D}t}
\end{aligned}
$$

对于不可压缩无黏性流体,其若体积力 $\vec{f}$ 仅由保守力提供,且密度恒定为 $\rho$ ,且流体满足定常条件即 $\frac{\partial}{\partial t} = 0$ ,则:

$$
\vec{v} \cdot \nabla(\frac{1}{2} v^2 + \frac{p}{\rho} + \Phi) = 0
$$

因此,在同一条流线上:

$$
\frac{1}{2} \rho v^2 + p + \rho\Phi = const
$$

若流体额外满足无旋条件,即 $\nabla \times \vec{v} = 0$ ,则此时,上式(即伯努利方程)对空间任意一处成立。

\section{后记}

感谢王海涛老师一年以来的教导! 由于篇幅限制,这里就不多展开了,更多的话请\href{https://darksharpness.github.io/haitallica/}{点击这里}。

感谢 \href{https://hastin-blog.cn/}{Hastin} 和 \href{https://github.com/hsfzLZH1}{hsfzLZH1} 同学在百忙之中抽出时间来帮忙审稿。

\begin{thebibliography}{30}

    \bibitem{ref0} 维基百科:物质导数\url{https://zh.wikipedia.org/wiki/%E7%89%A9%E8%B3%AA%E5%B0%8E%E6%95%B8} 
    \bibitem{ref1} 维基百科:雷诺传输定理\url{https://zh.wikipedia.org/wiki/%E9%9B%B7%E8%AB%BE%E5%82%B3%E8%BC%B8%E5%AE%9A%E7%90%86} 
    \bibitem{ref2} 维基百科:积分符号内取微分\url{https://zh.wikipedia.org/wiki/%E7%A7%AF%E5%88%86%E7%AC%A6%E5%8F%B7%E5%86%85%E5%8F%96%E5%BE%AE%E5%88%86#%E9%AB%98%E7%BB%B4%E6%83%85%E5%86%B5}
    \bibitem{ref3} 知乎:物理学家用流体力学 · 第二章:流体的应力与应变\url{https://zhuanlan.zhihu.com/p/554300210}
    % \bibitem{ref4} 维基百科:纳维-斯托克斯方程 \url{https://zh.wikipedia.org/wiki/%E7%BA%B3%E7%BB%B4-%E6%96%AF%E6%89%98%E5%85%8B%E6%96%AF%E6%96%B9%E7%A8%8B}


\end{thebibliography}

\end{document}
